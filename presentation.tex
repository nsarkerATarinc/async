\documentclass{beamer}
\usetheme{Rochester}
%\usecolortheme{albatross}

\usepackage[utf8]{inputenc}
\usepackage{amsmath}
\usepackage{amsfonts}
\usepackage{amssymb}

\AtBeginSection[]{
  \begin{frame}
  \vfill
  \centering
  \begin{beamercolorbox}[sep=8pt,center,shadow=false,rounded=false]{title}
    \usebeamerfont{title}\insertsectionhead\par%
  \end{beamercolorbox}
  \vfill
  \end{frame}
}

\author{nsarker@arinc.com}
\title{Asynchronous Programming}
\date{}


\begin{document}
\small

\begin{frame}
    \titlepage
\end{frame}


\begin{frame}
\frametitle{Programming Models: Sequential}
\begin{columns}[T]
    \begin{column}{0.1\textwidth}
        \includegraphics[scale=0.5]{images/sync.png}
    \end{column}
    \begin{column}{.5\textwidth}
        Synchronous or sequential execution occurs one after another.
    \end{column}
\end{columns}
\end{frame}


\begin{frame}
\frametitle{Programming Models: Threaded}
\begin{columns}[T]
    \begin{column}{.5\textwidth}
        \includegraphics[scale=0.4]{images/threaded.png}
    \end{column}
    \begin{column}{.5\textwidth}
        Synchronous or sequential execution occurs one after another.
    \end{column}
\end{columns}

\end{frame}


\begin{frame}
\frametitle{Programming Models: Asynchronous}
\begin{columns}[T]
    \begin{column}{.5\textwidth}
        \includegraphics[scale=0.4]{images/async.png}
    \end{column}
    \begin{column}{.5\textwidth}
        Asynchronous paradigm.
    \end{column}
\end{columns}

\end{frame}


\begin{frame}
\frametitle{Event Loop}
What is an \textit{event loop}, or \textit{reactor}? Think of them as task queues.

\begin{quote}
    The event loop is a programming construct that waits for and dispatches events or messages in a program. It works by calling some internal or external “event provider”, which generally blocks until an event has arrived, and then calls the relevant event handler (“dispatches the event”). - Twisted
\end{quote}

Chances are you've seen event loop syntax in the past:
\begin{quote}
    \begin{semiverbatim}
    - app.run("0.0.0.0", 8888)

    - loop.start()
    \end{semiverbatim}
\end{quote}

Libraries:

\begin{tabular}{l l l l l}
    Twisted & Tornado & asyncio & gevent & uvloop
\end{tabular}

\end{frame}


\begin{frame}
\frametitle{"Why can't I just use a while loop?"}
A simple while loop is very inefficient. Most modern async frameworks are built on top of \textit{poll/epoll} (Linux), \textit{kqueue} (BSD/MacOS), or \textit{select} (most operating systems).
\end{frame}


\begin{frame}
\frametitle{Deferreds, Coroutines, Futures}
\begin{columns}
    \begin{column}{0.45\textwidth}
        Deferred
        \begin{semiverbatim}
            d = Deferred()
            d.addCallback(f1)
            d.addCallback(f2)
            d.callback('result')
        \end{semiverbatim}
    \end{column}
    \begin{column}{0.35\textwidth}
        Futures
        \begin{semiverbatim}
            d = Futures()
            d.addCallback(f1)
            d.addCallback(f2)
            d.callback('result')
        \end{semiverbatim}
    \end{column}
    \begin{column}{0.5\textwidth}
        Coroutine
        \begin{semiverbatim}
            async def coroutine():
                d = await func()
                return d
        \end{semiverbatim}
    \end{column}
\end{columns}
\end{frame}


\begin{frame}
\frametitle{Asynchronous != Magic}
\end{frame}


\section{Asynchronous Web Development}
\begin{frame}
\frametitle{Breaking Up Work}
\end{frame}


\begin{frame}
\frametitle{External Requests}
\end{frame}


\begin{frame}
\frametitle{Database Connections}
\end{frame}


\begin{frame}
\frametitle{Websockets}
\end{frame}

\end{document}
